% !Mode:: "TeX:UTF-8"
% !TEX program  = xelatex
\documentclass[withoutpreface,bwprint]{cumcmthesis}  %样式模板引入

\usepackage{gbt7714} %引入标准引用格式样式文件

\usepackage{array}   %引入矩阵宏包

\usepackage{booktabs}%引入制表符宏包
\usepackage{witharrows} %带箭头公式宏包
\bibliographystyle{gbt7714-numerical}%引入中文期刊标准引用格式宏包
\title{Hello Equation}
\tihao{ABCD}            % 题号
\baominghao{114514}    % 报名号
\schoolname{CQUPT}
	\membera{Peitsan}
\memberb{SYgiegie}
\memberc{THgiegie}
\supervisor{Boss Duan}
\yearinput{2022}     % 年
\monthinput{09}      % 月
\dayinput{13}        % 日

\begin{document}
	\maketitle % 标题
	\begin{abstract}
		摘要的具体内容。摘要的具体内容。摘要的具体内容。摘要的具体内容。摘要的具体内容。摘要的具体内容。摘要的具体内容。摘要的具体内容。摘要的具体内容。摘要的具体内容。摘要的具体内容。摘要的具体内容。摘要的具体内容。摘要的具体内容。摘要的具体内容。摘要的具体内容。摘要的具体内容。摘要的具体内容。摘要的具体内容。摘要的具体内容。摘要的具体内容。摘要的具体内容。摘要的具体内容。摘要的具体内容。
		\keywords{关键词1\quad  关键词2\quad   关键词3}
	\end{abstract}
	\newpage
	
\section{认识数学公式Tex环境}
	
\noindent 公式块级标签环境


文字注释 文字注释 文字注释 文字注释 


%\begin{equation}
%	公式体
%\end{equation}

	\begin{equation}
			f(x)= \frac{\sum_{i=0}^{n}}{n} 
	\end{equation}
文字注释 文字注释 文字注释 文字注释 


	\begin{equation*}
		f(x)= \frac{\sum_{i=0}^{n}}{n} 
	\end{equation*}

\noindent 单行嵌入公式环境


%  $ 公式体 $

文字注释文字注释 文字注释    $ f(x)= \frac{\sum_{i=0}^{n}}{n} $ 文字注释 文字注释 文字注释 文字注释 

\noindent 多行行级公式环境 

文字注释 文字注释 文字注释 文字注释 
$$
  	 f(x)= \frac{\sum_{i=0}^{n}}{n} \\
		s(x)= \lim_{x\rightarrow\infty}\sum_{i=0}^{n}f(x)\\
		p(x)= \frac{s(x)}{\int_{x\rightarrow\infty}e^{x};}
$$
文字注释 文字注释 文字注释 

\section{公式序号管理}

\subsection{公式默认编号}

\noindent equation自动编号

	\begin{equation}
		f(x)= \frac{\sum_{i=0}^{n}}{n}
	\end{equation}

	\begin{equation}
		s(x)= \lim_{x\rightarrow\infty}\sum_{i=0}^{n}f(x)
	\end{equation}


\subsection{公式章节编号}

\noindent 按照章节手动编号()
%\tag{自定义编号}

	\begin{equation}
		\tag{2-2-1}
		f(x)= \frac{\sum_{i=0}^{n}}{n}
	\end{equation}
	\begin{equation}
	\tag{2-2-2}
	s(x)= \lim_{x\rightarrow\infty}\sum_{i=0}^{n}f(x)
\end{equation}

\section{简单公式编辑}

% 数字 常量 

% 四则运算

% 积分、高等代数

% 集合运算

% 逻辑代数

% 概率论 



	\subsection{希腊字母、特殊符号与常量、单位符号等}
	
	\noindent 0.希腊数字
	
	$ I $
	
		$ IV $
	
	\noindent 1.希腊字母
	
		$\alpha \quad \beta \quad \gamma \quad \lambda \quad \theta \quad \xi \quad \eta \quad \sigma \quad \chi \quad \psi \quad \phi \quad \mu \quad \varphi \quad \omega$
			
			
		$A \quad B \quad \Gamma \quad \Lambda \quad \Theta \quad \Xi \quad E \quad \Sigma \quad X \quad \Psi  \quad \Phi\quad V U \quad \quad \Omega$	
			
	\noindent 2.特殊符号与常量
	
	%转移关键字
	   * $\times$  $\cdot$ 
	   
	   ~ $\sim$ 
	   
%	   \ $\ $
	   
	   \&
%	   & 
	   
	   $\{ \}$
	   
	   $[]$ 
	   
	   \%
	   
	   \_
	    
	    
	    
	   / $\textdiv$
	 
	 \begin{equation}
	 	\begin{split}
	 	\vec{z} = (x,y) \\
	 	\|z\|  = \sqrt{x^{2}+y^{2}}
	 	\end{split}
	 \end{equation}
	 
		$\pi \quad \tau \quad \epsilon \quad \times \textdiv \quad \cdot \quad \sim \quad\surd \leq \quad \geq \quad\gtrapprox \quad \|X\|  $
	  
	  vector $\widehat{f(x)}$
	  
	  
	  $\cos \quad \sin \quad \textdiv \quad \arg  \quad\cdot \quad \lim \quad \ln \quad  \log \quad \hat{x} \quad \vec{x} \quad  \widehat{abc} \quad \tilde{a} \quad \dot{a}$
	  
	\noindent 3.单位符号
	
	
	$\textyen$
	
	$° \textdegree \quad ℃  \textcelsius$ 
	
	$ \textdegree \quad \textcelsius \quad \textohm \quad \S \quad \textpertenthousand \quad \textperthousand \quad \textyen \quad
	$
	\subsection{上下标、分数、积分、导数等}
	$
			\lambda^{e^{\epsilon_{p^{x}} }} \quad^ \lambda^{(x,y)} \quad
		x^{n}  \quad x_{k}  \quad x^{n}\_{k} \quad x^{n}_{k}  
	$
	
	$
			f(x) = \int_{a^{2}}}^{b}e^{x} dg(x)
	$
	
	$ 
		f'(x) \quad f''(x) \quad f'''(x) \quad f^{(n)}(x)
	$
 %提高	
%	\begin{equation}
%		\begin{split}
%			P(x_{1},x_{2}\cdots,x_{n})& =\prod_{i=1}^{n}p(x_{i}) \\ &=\frac{1}{\sigma\sqrt(2\pi^{\pi})}exp(-\frac{1}{2}\sum_{i=1}^{n}\frac{(x_{i}-\mu_{i})^2}{2\sigma^2_{i}})
%		\end{split}
%	\end{equation}


	\subsection{无理数、无穷、极限}
	
		$^{3}\sqrt{x^{2}} $
		
		
		infinity
		
		$ \sqrt{x}  \quad ^3\sqrt{x}\quad \infty $
		
		\begin{equation}
			\lim_{x\rightarrow\infty}
		\end{equation}
	
	\subsection{级数、重级数}
		\begin{equation}
			\ln{(1-x)} =\lim_{N\textrightarrow\infty} x+x^{2}+x^{3} \cdots x^{n} \cdots  = \lim_{N\textrightarrow\infty}\sum_{n=1}^{N} x^n
		\end{equation}
	
		\begin{equation}
		\sum_{n=1}^{N} \sum_{k=1}^{n} \frac{1}{x^{k}}
		\end{equation}
		
	\subsection{集合、概率、逻辑运算}
		$	
			\circ \quad \bullet \quad\subseteq \quad \supseteq \quad \supsetneqq \quad \subsetneqq \quad \quad \bigodot  \quad \oplus \quad \otimes  \quad \bigcap \quad \bigcup \quad \complement  \quad \bigwedge \quad \bigvee
		$
		 
		 
		 $ 
		  \{(x,y) | x<R \;,y<R\}
		  $
		  
		 $ 
		  A = \complement_{D}(B \bigcap E)
		  $
		  
		   $ 
		   	p(y^{*}\mid x^{*},X,Y) = \int p(y^{*}\mid f^{*})p(f^{*}\mid x^{*},X,Y)df^{*}
		   $ 
		   
		   
		 |  $\mid $ \\
		 
		   ||  $\| $\\
		   
		   
		   $ 
		   	AB \bigodot C \oplus D \bigwedge E \subseteq F
		   $ 
		   
\section{高阶公式编辑}


	\subsection{多行公式}
		\begin{equation}
			\begin{split}
				&\Gamma(w,b) = \frac{a}{\|W\|} 
				\\
				&Max_{(w,b)}  \Gamma 
				\\
				s.t \quad  y_{i}(w^{T}x_{i}+b) &\geq 1, \quad i =1,2,\cdots,m.
			\end{split}
		\end{equation}
	
	
	\subsection{大括号、矩阵}
	
	
		\begin{equation}
			\left\{
				\begin{split}
					x & = \rho+ r cos\theta \\
					y &= \rho + r sin\theta 	\\
					z &= \theta^{k}
				\end{split}
			\right
		\end{equation}
	
		\begin{equation}
			\left
			\begin{split}
				x & = \rho+ r cos\theta \\
				y &= \rho + r sin\theta 	\\
				z &= \theta^{k}
			\end{split}
			\right\}
		\end{equation}
	\begin{equation}
		\left\{
		\begin{split}
			x & = \rho+ r cos\theta \\
			y &= \rho + r sin\theta 	\\
			z &= \theta^{k}
		\end{split}
		\right\}
	\end{equation}
	\\
	
	colume vertical 
		\begin{equation}
			C(x,x')=\begin{bmatrix}
				c(x_{1},x'_{1}) &	c(x_{1},x'_{2})&\cdots&c(x_{1},x'_{n}) \\
				c(x_{2},x'_{1}) & c(x_{2},x'_{2})  &\cdots&c(x_{2},x'_{n})
				\\
				\vdots           & \vdots          &\ddots&\vdots 
				\\
				c(x_{n},x'_{1}) & c(x_{n},x'_{2})&\cdots&c(x_{n},x'_{n})	
			\end{bmatrix}
		\end{equation}
		
		\begin{equation}
			C(x,x')=\begin{matrix}
				c(x_{1},x'_{1}) &	c(x_{1},x'_{2})&\cdots&c(x_{1},x'_{n}) \\
				c(x_{2},x'_{1}) & c(x_{2},x'_{2})  &\cdots&c(x_{2},x'_{n})
				\\
				\vdots           & \vdots          &\ddots&\vdots 
				\\
				c(x_{n},x'_{1}) & c(x_{n},x'_{2})&\cdots&c(x_{n},x'_{n})	
			\end{matrix}
		\end{equation}
		
			\begin{equation}
			C(x,x')=\begin{vmatrix}
				c(x_{1},x'_{1}) &	c(x_{1},x'_{2})&\cdots&c(x_{1},x'_{n}) \\
				c(x_{2},x'_{1}) & c(x_{2},x'_{2})  &\cdots&c(x_{2},x'_{n})
				\\
				\vdots           & \vdots          &\ddots&\vdots 
				\\
				c(x_{n},x'_{1}) & c(x_{n},x'_{2})&\cdots&c(x_{n},x'_{n})	
			\end{vmatrix}
		\end{equation}
		\begin{equation}
		C(x,x')=\begin{pmatrix}
			c(x_{1},x'_{1}) &	c(x_{1},x'_{2})&\cdots&c(x_{1},x'_{n}) \\
			c(x_{2},x'_{1}) & c(x_{2},x'_{2})  &\cdots&c(x_{2},x'_{n})
			\\
			\vdots           & \vdots          &\ddots&\vdots 
			\\
			c(x_{n},x'_{1}) & c(x_{n},x'_{2})&\cdots&c(x_{n},x'_{n})	
		\end{pmatrix}
	\end{equation}
	\subsection{复杂逻辑带箭头}
	
	
	\begin{equation}
			\left 
		\begin{WithArrows}\Arrow{我们展开}
			A & = (a+1)^2  \\
			& = a^2 + 2a + 1 \\
			& = a^2 + 2a + 1
		\end{WithArrows}\right
	\end{equation}

	\subsection{带条件公式}
			\begin{equation}
					L=\mid F_{0}F_{k}\mid ,\quad k\in(0,10)	
			 \end{equation}
	\subsection{优化目标函数}
	
			\begin{equation}
			\begin{split}
				Loss_{k} & = \frac{(\rho_{k}' - \rho_{k})^{2}}{\sum_{k=2}^{9}(\rho_{k}' - \rho_{k})^{2}} + \frac{(\theta_{k}' - \theta_{k})^{2}}{\sum_{k=2}^{9}(\theta_{k}' - \theta_{k})^{2}}  \\
				&max\quad\sum_{k=2}^{9}Loss_{k}
			\end{split}
		\end{equation}	
	
	\subsection{长公式规范书写}
	
		\begin{equation}
		\tag{不规范}
		\begin{split}
			k'\_{i}&= \sum_{i}( \sum_{i}+ \sum_{i+1})^{-1}\\
			&=\frac{H_{k'\_{i}}P_{k'\_{i}}H_{k'\_{i}}^{T}}{H_{k'\_{i}}P_{k'\_{i}}H_{k'\_{i}}^{T}+R_{k'\_{i}}} \\
			U'&= U_{i}+ k'(U_{i+1}- U_{i}) \\
			\sum'& = \sum_{i} - k'\_{i}\sum_{i} \\
			P'\_{k'}& =P_{k}-k'H_{k'\_{i}}P_{k}
		\end{split}
	\end{equation}


		\begin{equation}
			\tag{规范}
			\begin{split}
				k'\_{i}&= \sum_{i}( \sum_{i}+ \sum_{i+1})^{-1}
				\\
				       &=\frac{H_{k'\_{i}}P_{k'\_{i}}H_{k'\_{i}}^{T}}{H_{k'\_{i}}P_{k'\_{i}}H_{k'\_{i}}^{T}+R_{k'\_{i}}} \\
			      	U'&= U_{i}+ k'(U_{i+1}- U_{i}) 
			      	\\
				\sum'& = \sum_{i} - k'\_{i}\sum_{i}
				 \\
			  P'_{k'}& =P_{k}-k'H_{k'_{i}}P_{k}
			\end{split}
		\end{equation}
	
	$X_{y=0}$
	\bibliography{bib/cite.bib}
	\newpage
	
\end{document}
