\documentclass[normalsize]{article}   %样式模板
%\documentclass[默认字号]{article}   %修改默认字号

\usepackage{subfigure} %引入图片插入宏包
\usepackage{booktabs}
\usepackage[graphicx]{realboxes} %引入图片插入宏包
%\includegraphics[]{}


\usepackage{ctex}  %切换为中文
\usepackage{caption} %caption标题宏包
%opening
\title{Hello TexGrammar}  %文章标题
\author{Peitsan}    %文章作者
\begin{document}  %Tex体
	\maketitle
	\begin{abstract}  %摘要
		这是摘要这是摘要这是摘要这是摘要这是摘要这是摘要这是摘要这是摘要这是摘要这是摘要这是摘要这是摘要这是摘要这是摘要这是摘要这是摘要这是摘要	这是摘要这是摘要这是摘要这是摘要这是摘要这是摘要这是摘要这是摘要这是摘要这是摘要这是摘要这是摘要这是摘要这是摘要这是摘要这是摘要这是摘要	这是摘要这是摘要这是摘要这是摘要这是摘要这是摘要
		这是摘要这是摘要这是摘要这是摘要这是摘要这是摘要这是摘要这是摘要这是摘要这是摘要这是摘要这是摘要这是摘要这是摘要这是摘要这是摘要这是摘要	这是摘要这是摘要这是摘要这是摘要这是摘要这是摘要这是摘要这是摘要这是摘要这是摘要这是摘要这是摘要这是摘要这是摘要这是摘要这是摘要这是摘要	这是摘要这是摘要这是摘要这是摘要这是摘要这是摘要这是摘要这是摘要这是摘要这是摘要这是摘要这是摘要这是摘要这是摘要这是摘要这是摘要这是摘要	这是摘要这是摘要这是摘要这是摘要这是摘要这是摘要这是摘要这是摘要这是摘要这是摘要这是摘要这是摘要这是摘要这是摘要这是摘要这是摘要这是摘要	这是摘要这是摘要这是摘要这是摘要这是摘要这是摘要这是摘要这是摘要这是摘要这是摘要这是摘要这是摘要这是摘要这是摘要这是摘要这是摘要这是摘要
	\end{abstract}

\newpage

	\tableofcontents %这是目录
	
\newpage
	\section{盒模型}
	 
	  	\subsection{双栏盒子}
	  	
		\begin{figure}[htbp!]
			\begin{minipage}{0.5\linewidth}
				%盒子高度继承父级元素的高度
				\centering    %让盒内元素左右居中对齐 默认上下居中停靠
				\includegraphics[height=5cm]{img/sample.png}
				%\includegraphics[命令]{图片路径}
				\caption{狗头1狗头1狗头1狗头1狗头1狗头1狗头1狗头1狗头1狗头1狗头1狗头1狗头1狗头1狗头1狗头1狗头1狗头1}
				\label{fig:1}  %图片标题
			\end{minipage}
			\begin{minipage}{0.5\linewidth}
				\centering
				\includegraphics[height=2cm]{img/sample.png}
				\caption{狗头1狗头1狗头1狗头1狗头1狗头1狗头1狗头1狗头1狗头1狗头1狗头1狗头1狗头1狗头1狗头1狗头1狗头1狗头1狗头1狗头1狗头1狗头1狗头1}
\label{fig:1}  %图片标题
			\end{minipage}
		\end{figure}
	
		\subsection{三栏盒子}
		
		\begin{figure}[htbp!]
			\begin{minipage}{0.32\linewidth}
				\centering    
				\includegraphics[height=4cm]{img/sample.png}
			\end{minipage}
			\begin{minipage}{0.32\linewidth}
				\centering    
				\includegraphics[height=2.5cm]{img/sample.png}
			\end{minipage}
			\begin{minipage}{0.32\linewidth}
				\centering
				\includegraphics[height=3cm]{img/sample.png}
			\end{minipage}
		\end{figure}
		
		\begin{figure}[htbp!]
			\begin{minipage}{0.32\linewidth}
				\centering    
				\includegraphics[height=3cm]{img/sample.png}
			\end{minipage}
			\begin{minipage}{0.1\linewidth}
				\centering    
				\includegraphics[height=3cm]{img/sample.png}
			\end{minipage}
			\begin{minipage}{0.1\linewidth}
				\centering
				\includegraphics[height=3cm]{img/sample.png}
			\end{minipage}
		\end{figure}
\newpage	
	\section{Tex中文本的操作编辑}
	
		\subsection{缩进与空格}
		
			 \noindent
			 1. Tex中段落默认首行缩进 
			
			\noindent 
			2.取消首行缩进:
			
			在段落最前面添加 \verb|\|noindent 
			
			\noindent \textbf{这是不添加去缩进的状态,文段默认退两格}
			
			这是一段话这是一段话这是一段话这是一段话这是一段话这是一段话这是一段话这是一段话这是一段话这是一段话这是一段话这是一段话这是一段话这是一段话这是一段话这是一段话这是一段话这是一段话这是一段话这是一段话这是一段话这是一段话这是一段话。
			
			\noindent \textbf{这是添加去缩进的状态,文段顶格。}
			
			\noindent 这是一段话这是一段话这是一段话这是一段话这是一段话这是一段话这是一段话这是一段话这是一段话这是一段话这是一段话这是一段话这是一段话这是一段话这是一段话这是一段话这是一段话这是一段话这是一段话这是一段话这是一段话这是一段话这是一段话。
			
			
			eg.
			
			 这是一段话这是一段话这是一段话\qquad 这是一段\quad 话这是一段话这是一段话这是一段话这是一段话这是一段话这是一段话这是一段话这是一段话这是一段话这是一段话这是一段话这是一段话这是一段话这是一段话这是一段话这是一段话这是一段话这是一段话这是一段话。 这是一段话这是一段话这是一段话这是一段话这是一段话这是一段话这是一段话这是一段话这是一段话这是一段话这是一段话这是一段话这是一段话这是一段话这是一段话这是一段话这是一段话这是一段话这是一段话这是一段话这是一段话这是一段话这是一段话。 这是一段话这是一段话这是一段话这是一段话这是一段话这是一段话这是一段话这是一段话这是一段话这是一段话这是一段话这是一段话这是一段话这是一段话这是一段话这是一段话这是一段话这是一段话这是一段话这是一段话这是一段话这是一段话这是一段话。 这是一段话这是一段话这是一段话这是一段话这是一段话这是一段话这是一段话这是一段话这是一段话这是一段话这是一段话这是一段话这是一段话这是一段话这是一段话这是一段话这是一段话这是一段话这是一段话这是一段话这是一段话这是一段话这是一段话。 这是一段话这是一段话这是一段话这是一段话这是一段话这是一段话这是一段话这是一段话这是一段话这是一段话这是一段话这是一段话这是一段话这是一段话这是一段话这是一段话这是一段话这是一段话这是一段话这是一段话这是一段话这是一段话这是一段话。	
			
      \noindent 3.各种空格宽度对比:
      
		\noindent\verb|\|qquad 
		
		  \noindent	这是一行话这是一行话 \qquad 这是一行话\\
		 \verb|\|quad 
		 
		  	 \noindent	这是一行话这是一行话 \quad 这是一行话\\
		  \verb|\| 
		  	  	
		  	 \noindent	这是一行话这是一行话\ 这是一行话\\
		  \verb|\|;
		  
		  	 \noindent	这是一行话这是一行话\;这是一行话\\
		  \verb|\|, 
		  	
		  	 \noindent	这是一行话这是一行话\,这是一行话\\
		  \verb|\|!	
		  
		  	 \noindent	这是一行话这是一行话\!这是一行话\\
		  	
		\subsection{加粗、斜体、下划线}
		
		\noindent 1.加粗: \textbf{xxxxxxxxxxxxxxxx} broadenfonts
		
		\textbf{这是一段话}这是一段话这是一段话这是一段话\textbf{这是一段加粗的话这是一段加粗的话这是一段加粗的话这是一段加粗的话}这是一段话这是一段话这是一段话\textbf{这是一段加粗的话这是一段加粗的话这是一段加粗的话这是一段加粗的话}这是一段话这是一段话这是一段话这是一段话这是一段话这是一段话这是一段话这是一段话。
		
		\noindent 2.斜体:\textit{加斜体}
		
			\textit{XXXXXXX}这是一段话这是一段话这是一段话\textit{这是一段斜体的话这是一段加粗的话这是一段斜体的话这是一段斜体的话}这是一段话这是一段话这是一段话\textit{这是一段斜体的话这是一段加粗的话这是一段斜体的话这是一段斜体的话}这是一段话这是一段话这是一段话这是一段话这是一段话这是一段话这是一段话这是一段话。
		
		\noindent 3.下划线:
		
			这是一段话这是一段话这是一段话这是一段话\underline{这是一段下划线的话这是一段下划线的话}这是一段话这是一段话这是一段话\underline{这是一段下划线的话}这是一段话这是一段话这是一段话这是一段话这是一段话这是一段话这是一段话这是一段话。
			
		\noindent 4.正文仿宋:
		
		这是一段话这是一段话这是一段话这是一段话\texttt{这是一段正文仿宋这是一段正文仿宋这是一段正文仿宋这是一段正文仿宋}这是一段话这是一段话这是一段话\texttt{这是一段正文仿宋这是一段正文仿宋这是一段正文仿宋这是一段正文仿宋}这是一段话这是一段话这是一段话这是一段话这是一段话这是一段话这是一段话这是一段话。
		
		\subsection{修改字号}
		
		\noindent 1.标题字号:
			
					\begin{figure}[htbp!]
						\centering
						\captionsetup{font={small}}%修改标题字号
						\includegraphics[height=4cm]{./img/sample.png}
						\caption{是一个狗头}
						\label{fig:10}
			    	\end{figure}
		    	
		\noindent 2.正文字号: 小正常
		
		正常大小正常大小\small{small大小small大小small大小small大小small大小small大小small大小small大小}正常大小正常大小正常大小正常\tiny{tiny大小tiny大小tiny大小tiny大小tiny大小tiny大小tiny大小}大小正常大小正常大小正常大小正常大小正常大小正常\huge{huge大小huge大小huge大小huge大小huge大小}
		
		
		 \Huge Huge大小Huge大小Huge大小Huge大小Huge大小Huge大小Huge大小
		 \normalsize
		 Huge大小Huge大小Huge大小Huge大小Huge大小Huge大小Huge大小\\
		 
		 
		 Huge大小Huge大小Huge大小Huge大小Huge大小Huge大小Huge大小
		 \\
		 
				
		\normalsize
		\newpage
	\section{Tex中图片的操作编辑}
		\subsection{引入一张图片}
				% 单位 cm  像素px
		
%				\includegraphics[width=7cm,height=1cm]{图片}
		
%				\includegraphics[width=7cm]{./cover/img2/sample.png}
		
			\begin{figure}[htbp!]
				\centering
				\captionsetup{font={footnotesize}}%修改标题字号
				\includegraphics[height=4cm,width=4cm]{./img/sample.png}%图片长宽设置为4cm,图片路径在img文件下(记得带上文件类型名)
				\caption{是一个狗头是一个狗头是一个狗头是一个狗头是一个狗头是一个狗头}
				\label{fig:10}
			\end{figure}
		
		
		\newpage
		\subsection{图片的浮动}
			这是一段话这是一段话这是一段话这是一段话这是一段话这是一段话这是一段话这是一段话这是一段话这是一段话这是一段话这是一段话这是一段话这是一段话这是一段话这是一段话这是一段话这是一段话这是一段话这是一段话这是一段话这是一段话这是一段话。
		\begin{figure}[h]
			\centering
			\captionsetup{font={small}}%修改标题字号
			\includegraphics[height=4cm,width=4cm]{./img/sample.png}%图片长宽设置为4cm,图片路径在img文件下(记得带上文件类型名)
			\caption{是一个狗头}
			\label{fig:10}
		\end{figure}
		这是一段话这是一段话这是一段话这是一段话这是一段话这是一段话这是一段话这是一段话这是一段话这是一段话这是一段话这是一段话这是一段话这是一段话这是一段话这是一段话这是一段话这是一段话这是一段话这是一段话这是一段话这是一段话这是一段话。
		\newpage
			这是一段话这是一段话这是一段话这是一段话这是一段话这是一段话这是一段话这是一段话这是一段话这是一段话这是一段话这是一段话这是一段话这是一段话这是一段话这是一段话这是一段话这是一段话这是一段话这是一段话这是一段话这是一段话这是一段话。
			\begin{figure}[t]
			\centering
			\captionsetup{font={small}}%修改标题字号
			\includegraphics[height=4cm,width=4cm]{./img/sample.png}%图片长宽设置为4cm,图片路径在img文件下(记得带上文件类型名)
			\caption{是一个狗头}
			\label{fig:10}
		\end{figure}
		这是一段话这是一段话这是一段话这是一段话这是一段话这是一段话这是一段话这是一段话这是一段话这是一段话这是一段话这是一段话这是一段话这是一段话这是一段话这是一段话这是一段话这是一段话这是一段话这是一段话这是一段话这是一段话这是一段话。
		\newpage
			这是一段话这是一段话这是一段话这是一段话这是一段话这是一段话这是一段话这是一段话这是一段话这是一段话这是一段话这是一段话这是一段话这是一段话这是一段话这是一段话这是一段话这是一段话这是一段话这是一段话这是一段话这是一段话这是一段话。
%			bottom
			\begin{figure}[b]
			\centering
			\captionsetup{font={small}}%修改标题字号
			\includegraphics[height=4cm,width=4cm]{./img/sample.png}
			\caption{是一个狗头}
			\label{fig:10}
		\end{figure}
	这是一段话这是一段话这是一段话这是一段话这是一段话这是一段话这是一段话这是一段话这是一段话这是一段话这是一段话这是一段话这是一段话这是一段话这是一段话这是一段话这是一段话这是一段话这是一段话这是一段话这是一段话这是一段话这是一段话。
	
	
		这是一段话这是一段话这是一段话这是一段话这是一段话这是一段话这是一段话这是一段话这是一段话这是一段话这是一段话这是一段话这是一段话这是一段话这是一段话这是一段话这是一段话这是一段话这是一段话这是一段话这是一段话这是一段话这是一段话。
		
		
		
			这是一段话这是一段话这是一段话这是一段话这是一段话这是一段话这是一段话这是一段话这是一段话这是一段话这是一段话这是一段话这是一段话这是一段话这是一段话这是一段话这是一段话这是一段话这是一段话这是一段话这是一段话这是一段话这是一段话。
			
			
				这是一段话这是一段话这是一段话这是一段话这是一段话这是一段话这是一段话这是一段话这是一段话这是一段话这是一段话这是一段话这是一段话这是一段话这是一段话这是一段话这是一段话这是一段话这是一段话这是一段话这是一段话这是一段话这是一段话。
	
		\newpage
			这是一段话这是一段话这是一段话这是一段话这是一段话这是一段话这是一段话这是一段话这是一段话这是一段话这是一段话这是一段话这是一段话这是一段话这是一段话这是一段话这是一段话这是一段话这是一段话这是一段话这是一段话这是一段话这是一段话。
			\begin{figure}[p]
			\centering
			\captionsetup{font={small}}%修改标题字号
			\includegraphics[height=4cm,width=4cm]{./img/sample.png}%图片长宽设置为4cm,图片路径在img文件下(记得带上文件类型名)
			\caption{是一个狗头}
			\label{fig:10}
		\end{figure}
		这是一段话这是一段话这是一段话这是一段话这是一段话这是一段话这是一段话这是一段话这是一段话这是一段话这是一段话这是一段话这是一段话这是一段话这是一段话这是一段话这是一段话这是一段话这是一段话这是一段话这是一段话这是一段话这是一段话。
		
		
		
		
		
		\newpage
		
		这是一段话这是一段话这是一段话这是一段话这是一段话这是一段话这是一段话这是一段话这是一段话这是一段话这是一段话这是一段话这是一段话这是一段话这是一段话这是一段话这是一段话这是一段话这是一段话这是一段话这是一段话这是一段话这是一段话。
		
			\begin{figure}[htbp!]
			\centering
			\captionsetup{font={small}}%修改标题字号
			\includegraphics[height=4cm,width=4cm]{./img/sample.png}%图片长宽设置为4cm,图片路径在img文件下(记得带上文件类型名)
			\caption{是一个狗头}
			\label{fig:10}
		\end{figure}
	
	这是一段话这是一段话这是一段话这是一段话这是一段话这是一段话这是一段话这是一段话这是一段话这是一段话这是一段话这是一段话这是一段话这是一段话这是一段话这是一段话这是一段话这是一段话这是一段话这是一段话这是一段话这是一段话这是一段话。
		\newpage
	\section{Tex中表格的操作编辑}
	
	\subsection{center环境下的表格}
	
		\begin{center}
			\begin{tabular}{cr}
				\hline
				\makebox[0.45\textwidth][l]{符号}	& \makebox[0.45\textwidth][r]{意义}   \\ \hline
				Symbol  & Meanings \\ 
				Symbol  & Meanings \\ \hline
				Symbol  & Meanings \\ 
				Symbol  & Meanings \\ 
				Symbol  & Meanings \\ \hline
			\end{tabular}
		\end{center}
	
	
	\subsection{table环境下的表格}
		\begin{table}[htbp!]
			\centering
			\begin{tabular}{crc}
				\hline
				\makebox[0.25\textwidth][c]{符号}	& \makebox[0.35\textwidth][c]{意义} & \makebox[0.2\textwidth][c]{单位}	 \\ \hline
				Symbol  & Meanings & Units\\ 
				Symbol  & Meanings & Units\\ \hline
			\end{tabular}
		\end{table}
		\subsection{table*环境下的表格}
			\begin{table*}[htbp!]
				\centering
				\begin{tabular}{ccc}
					\hline
					\makebox[0.25\textwidth][c]{符号}	& \makebox[0.35\textwidth][c]{意义} & \makebox[0.2\textwidth][c]{单位}	 \\ \hline
					Symbol  & MeaningsMeaningsMeaningsMeanings & Units\\
					\hline
				\end{tabular}%
				\label{tab:addlabel}%
			\end{table*}%
		\subsection{设置表格列宽}
		
		
			\begin{table}[htbp!]
				\tabcolsep=0.1cm %设置最小列间距 column
%				\tabcolsep=0.1cm %设置最小列间距
			\begin{tabular}{ccc}
				\hline
				\makebox[0.25\textwidth][c]{符号}	& \makebox[0.35\textwidth][c]{意义} & \makebox[0.2\textwidth][c]{单位}	 \\ \hline
				Symbol  & MeaningsMeaningsMeaningsMeanings & Units\\ \hline
				Symbol  & MeaningsMeaningsMeaningsMeanings & Units\\ \hline
			\end{tabular}
			\end{table}
		\subsection{设置表格行间距宽}
		
		\begin{table}[htbp!]
			\renewcommand\arraystretch{2} %设置行间距
			\begin{tabular}{ccc}
				\hline
				\makebox[0.25\textwidth][c]{符号}	& \makebox[0.35\textwidth][c]{意义} & \makebox[0.2\textwidth][c]{单位}	 \\ \hline
				Symbol  & Meanings & Units\\ \hline
				Symbol  & Meanings & Units\\ \hline
			\end{tabular}
		\end{table}
	
	% Please add the following required packages to your document preamble:
	% \usepackage{booktabs}
	\begin{table}[]
		\begin{tabular}{@{}lr@{}}
			\toprule
			\multicolumn{1}{c}{\textbf{符号}} & \multicolumn{1}{c}{意义} \\ \midrule
			1                               & 1                      \\
			1                               & 1                      \\
			1                               & 1                      \\
			1                               & 1                      \\ \bottomrule
		\end{tabular}
	\end{table}
\newpage

\appendix 
	\renewcommand{\appendixname}{Appendix~\Alph{section}}
	\section{附表1——XXXXXX}
	
\newpage	
\begin{thebibliography}{9}%宽度9
	\bibitem{bib:one}
	\bibitem{bib:two}
\end{thebibliography}

\end{document}